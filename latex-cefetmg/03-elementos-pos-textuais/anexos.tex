% -----------------------------------------------------------------------------
% Anexos
% -----------------------------------------------------------------------------


\begin{anexosenv}
\partanexos


% -----------------------------------------------------------------------------
% Primeiro anexo
% -----------------------------------------------------------------------------

\chapter{Nome do anexo}     % edite para alterar o título deste anexo
\label{chap:anexoA}

Lembre-se que a diferença entre apêndice e anexo diz respeito à autoria do texto e/ou material ali colocado.

Caso o material ou texto suplementar ou complementar seja de sua autoria, então ele deverá ser colocado como um apêndice. Porém, caso a autoria seja de terceiros, então o material ou texto deverá ser colocado como anexo.

Caso seja conveniente, podem ser criados outros anexos para o seu trabalho acadêmico. Basta recortar e colar este trecho neste mesmo documento. Lembre-se de alterar o "label"{} do anexo.

Organize seus anexos de modo a que, em cada um deles, haja um único tipo de conteúdo. Isso facilita a leitura e compreensão para o leitor do trabalho. É para ele que você escreve.


% -----------------------------------------------------------------------------
% Novo anexo
% -----------------------------------------------------------------------------

\chapter{Dica: nomes no BibTeX}
\label{chap:anexoB}

Reproduzo neste anexo, \textit{ipsis litteris}, o texto de autoria de \citeonline{Queiroz2014}.

Se você utiliza LaTeX para a redação de artigos já deve ter se deparado com algum tipo de problema no modo como o nome dos autores é apresentado no documento final (pior é quando a "descoberta"{} ocorre depois de já ter submetido o paper). Muitas vezes é difícil encontrar uma maneira certa de escrever o nome no arquivo *.bib e garantir que ele seja transcrito corretamente independente do estilo utilizado. Este texto tem o intuito de discutir o modo como o BibTeX interpreta o nome dos autores e ajudar na árdua tarefa de organizar a bibliografia.

Pessoalmente eu prefiro fornecer o nome completo dos meus autores para o BibTeX, sem abreviações e sem omitir nomes, quando possível. Desse modo, eu dou garantia que a minha bibliografia irá conter todos os dados para referenciar o autor independente do estilo utilizado para apresentá-lo. Depois disso, eu simplesmente espero que o BibTeX faça a abreviação e a colocação dos nomes da maneira correta de acordo com o estilo indicado. No entanto, para que essa tarefa seja feita é preciso apresentar os nomes da maneira correta para que a sua divisão seja feita de forma apropriada.

Para entender como o BibTeX divide um nome, é preciso conhecer antes as diversas partes que podem compor o nome de uma pessoa, que, a princípio, são: primeiro nome, nome do meio, ligação, último nome e júnior. A descrição de cada uma dessas partes é feita a seguir.

\begin{compactitem}
    \item \textbf{Primeiro nome:} é o nome da pessoa, geralmente utilizado para identificar uma pessoa em um contexto informal. Ex.: Diego, João, Maria etc. Em alguns casos o primeiro nome pode ser composto por dois nomes, como Maria Ana, Victor Hugo, etc. Nestes casos, deve-se observar como a pessoa utiliza o nome para poder diferenciar a segunda parte como Primeiro nome ou Nome do meio.

    \item \textbf{Nome do meio:} é o nome que sucede o primeiro nome, mas antecede o último nome, geralmente abreviado, por simplicidade. Ex.: Alan Mathison Turing, "Mathison"{} é o nome do meio. É comum uma pessoa possuir mais do que um nome do meio e também é comum que o nome do meio de alguns autores seja desconhecido, devido às abreviações e omissões feitas pelo mesmo.

    \item \textbf{Ligação:} também chamado de separador, são as palavras "de"{}, "da"{}, "do"{}, "e"{}, "von"{}, entre outras que ligam um nome ao outro. Em John von Neumann e Ricardo Luis de Azevedo da Rocha, por exemplo, as palavras "von"{}, "de"{}  e "da"{}  são as ligações. Num contexto geral, elas normalmente são grafadas com inicial minúscula para não serem confundidas com o nome do meio e, embora não seja comum em todos lugares do mundo, no Brasil é comum um nome possuir até mais do que uma ligação.

    \item \textbf{Último nome:} também chamado de nome de família, é o nome utilizado para identificar uma pessoa em situações formais, como referência em artigos, livros etc. Ex.: Albert Einstein, "Einstein"{} é o último nome.

\item \textbf{Júnior:} é um sufixo do nome que indica a existência de um parente com o mesmo nome. Geralmente abreviado como "Jr."{} pode ser apresentado de diversas formas como "Filho"{}, "Neto"{} ou traduzido para o idioma de origem do dono do nome, como "fils"{} (filho) em francês. Ex.: John Forbes Nash Jr.
\end{compactitem}

Quando indicamos o nome de um autor no BibTeX ele interpreta os nomes seguindo uma das três regras a seguir:

\begin{compactenum}
    \item \textbf{Nenhuma vírgula:} {Primeiro nome} {ligação} {Último nome}

    \item \textbf{Uma vírgula:} {ligação} {Último nome}, {Primeiro nome}

    \item \textbf{Duas vírgulas:} {ligação} {Último nome}, {Júnior}, {Primeiro nome}
\end{compactenum}

Como pode-se notar, a distinção entre essas três possíveis interpretações se dá com base na quantidade de vírgulas que foram inseridas e no posicionamento da ligação, que devem sempre ser escritas com a inicial minúscula. O(s) nome(s) do meio são todos os nomes que estão após o primeiro nome, porém antes da ligação e do último nome. A princípio, o BibTeX interpreta os nomes do meio como sendo parte do primeiro nome.

Para mostrar como isso pode gerar problemas, imagine, por exemplo, se o nome "John Forbes Nash Jr."{} fosse apresentado em um arquivo BibTeX. Como nenhuma vírgula foi inserida, será entendido que "John Forbes Nash"{} é o primeiro nome e "Jr."{} é o último nome, o que não seria correto. De forma semelhante, se for apresentado na forma "Nash Jr., John Forbes", então "John Forbes"{} será o primeiro nome enquanto "Nash Jr."{} será o último nome, que também está incorreto.

Portanto, a maneira correta de referenciar seria utilizando a terceira opção pois é a única que inclui o Jr. (utilizando duas vírgulas): "Nash, Jr., John Forbes"{}, fazendo com que "John Forbes"{} seja compreendido como primeiro nome, "Nash"{} como último nome e "Jr."{} como o júnior.

Outro grande problema ocorre quando um nome possui mais do que uma ligação, como em "Ricardo Luis de Azevedo da Rocha"{}. Quando o BibTeX lê um nome como esse, ele entende que tudo que vem após o ligador, faz parte do último nome. Neste caso, "Ricardo Luis"{} seria tratado como o primeiro nome e "Azevedo da Rocha"{} como último nome.

Para evitar esse comportamento, devemos optar pela segunda opção (utilizando uma vírgula), ou seja, "da Rocha, Ricardo {Luis de} Azevedo"{}, fazendo com que o último nome seja somente "Rocha"{} e precedido pelo seu ligador.

Note que neste último exemplo o ligador e o nome que o antecede foram delimitados por chaves. Este é um pequeno e útil truque que pode ser feito para garantir que os ligadores não sejam inclusos ao abreviar nomes (Ex.: Universidade de São Paulo, abrevia-se U.S.P. ao invés de U. de S.P. ou U.d.S.P.). Fazendo isso, o BibTeX passa a tratar "Luis de"{} como um único nome e o abrevia corretamente quando necessário.

E qual a importância de garantir que o BibTeX interprete corretamente as diversas partes de um nome? A verdade é que cada estilo trata o nome de uma maneira diferente: o IEEE, por exemplo, coloca apenas as iniciais do primeiro nome e a ligação seguida do último nome; a Nature, por outro lado, coloca a ligação e o último nome, seguido das iniciais do primeiro nome; e assim por diante. Assim sendo, entender como os nomes são interpretados nos ajuda a garantir que o mesmo seja sempre dividido da maneira correta e formatado apropriadamente independente do estilo fornecido.

Por fim, e não menos importante, também deixo aqui um aviso sobre a acentuação no BibTeX. Eu já presenciei diversos problemas com relação a acentuação nos nomes dos autores, títulos dos artigos etc. Em especial os problemas ocorreram quando eu estava utilizando o abnTeX, que é um projeto que tem o objetivo de implementar o padrão ABNT em formato TeX. Embora este projeto não seja um dos mais ativos, ele ainda é muito utilizado e alguns grupos de pesquisa utilizam estilos que nada mais são do que versões derivadas deste (como é o caso do laboratório que faço parte).

O problema é que este estilo possui uma falha (descrita em \href{http://abntex.codigolivre.org.br/node5.html}{http://abntex.codigolivre.org.br}), que impede que acentos sejam convertidos corretamente em letras maiúsculas. Para contornar o problema eles pedem que sejam utilizados códigos para descrever os acentos nos arquivos *.bib ao invés de inseri-los diretamente pelo teclado. Dado a quantidade de problemas que essa falha me gerou, julgo isso como uma boa prática e deixo aqui a minha recomendação de que não sejam utilizados caracteres não-ASCII nos arquivos *.bib.

Como os arquivos *.bib são interpretados pelo LaTeX, é possível utilizar alguns comandos em seus campos. A saber, segue os comandos para formar os acentos mais comuns:
\\
\\
\\



[A parte final do texto original foi suprimida, por conter incorreções.]\footnote{Nesta parte era apresentado os comandos \LaTeX{} para acentuação. No entanto, foi constatado que os comandos, se utilizados como apresentado, provocariam erros na transformação de  minúsculas para maiúsculas e vice-versa, algo bastante recorrente no estilo \texttt{abntex2}. Para a tabela com os comandos corretos veja \autoref{fig:acentos-latex}.}.
\\
\\
\\

[Em \href{http://en.wikibooks.org/wiki/LaTeX/Special_Characters}{http://en.wikibooks.org/wiki/LaTeX/Special\underline{ }Characters}] você encontra diversos outros acentos e símbolos para serem utilizados no LaTeX.


Referência:\\

Alexander Binder. Help On BibTeX Names. Disponível em <\href{www.kfunigraz.ac.at/~binder/texhelp/bibtx-23.html}{www.kfunigraz.ac.at/...}>. Acessado em 4 de março de 2011.


\end{anexosenv}
