% ------------------------------------------------------------------------------
% Introdução
% ------------------------------------------------------------------------------

\chapter{Introdução}
\label{chap_introducao}

A introdução deverá apresentar uma visão de conjunto do trabalho a ser realizado, com o apoio da literatura, situando-o no contexto do estado da arte da área científica específica, sua relevância no contexto da área inserida e sua importância específica para o avanço do conhecimento.

É uma boa prática iniciar cada novo capítulo com uma breve texto introdutório (tipicamente, dois ou três parágrafos) que deve deixar claro o quê será discutido no capítulo, bem como a organização do capítulo.
Também servirá ao propósito de ``amarrar'' ou ``alinhavar'' o conteúdo deste capítulo com o conteúdo do capítulo imediatamente anterior.

\section{Leia esta seção antes de começar}
\label{sec_leia_esta_secao_antes_de_omecar}

Este documento é um \emph{template} que foi concebido, primariamente, para ser utilizado na redação de teses de doutorado, dissertações de mestrado, projetos de qualificação tanto de mestrado quanto de doutorado, escritos em português brasileiro (eventualmente, com partes em inglês) e em conformidade com as normas da ABNT.

Não obstante, ele também poderá ser utilizado, com ligeiras adaptações para a redação de outros trabalhos acadêmicos monográficos (e.g., trabalhos de conclusão de curso de graduação ou de especialização \emph{lato sensu}).

Antes de começar a escrever o seu trabalho acadêmico utilizando este \emph{template}, é bom saber que há um arquivo principal que você precisará editar para que a capa e a folha de rosto de seu trabalho sejam geradas corretamente.
Este arquivo é o {\ttfamily meu-trabalho.tex}.
Nele, você deverá informar o seu nome, título do trabalho acadêmico, se o documento será uma tese de doutorado, dissertação de mestrado ou projeto de qualificação, nome de seu(s) orientador(es), e outras informações necessárias.
Neste arquivo, você deverá apenas comentar as linhas que não se aplicam ao seu tipo de trabalho acadêmico.
Todos os arquivos deste \emph{template} arquivos são auto-explicativos.

Para a compilar o documento, você pode utilizar o arquivo {\ttfamily makefile} por meio do comando {\ttfamily make}, disponível na mesma pasta onde está o arquivo principal {\ttfamily meu-trabalho.tex}.
No entanto atente para o fato de que, se você alterar o nome do arquivo {\ttfamily meu-trabalho.tex}, deverá também editar o arquivo {\ttfamily makefile} para alterá-lo do mesmo modo.

Por fim, caso observe algum problema ou qualquer outro tipo de falha ou mal comportamento neste modelo, comunique-nos para que possamos tentar corrigi-los em futuras atualizações.

\section{Justificativa}
\label{sec_justificativa}

Inserir seu texto aqui...

\section{Motivação}
\label{sec_motivacao}

Inserir seu texto aqui...

\section{Organização do trabalho}
\label{sec_organizacao_trabalho}

Normalmente ao final da introdução é apresentada, em um ou dois parágrafos curtos, a organização do restante do trabalho acadêmico.
Deve-se dizer o quê será apresentado em cada um dos demais capítulos.
