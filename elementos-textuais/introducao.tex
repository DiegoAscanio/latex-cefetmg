% ------------------------------------------------------------------------------
% Introdução
% ------------------------------------------------------------------------------

\chapter{Introdução}
\label{chap_introducao}

A introdução deverá apresentar uma visão de conjunto do trabalho a ser realizado, com o apoio da literatura, situando-o no contexto do estado da arte da área científica específica, sua relevância no contexto da área inserida e sua importância específica para o avanço do conhecimento.

É uma boa prática iniciar cada novo capítulo com um breve texto introdutório (tipicamente, dois ou três parágrafos) que deve deixar claro o quê será discutido no capítulo, bem como a organização do mesmo.
Também servirá ao propósito de ``amarrar'' ou ``alinhavar'' o conteúdo deste capítulo com o conteúdo do capítulo imediatamente anterior.

\section{Motivação}
\label{sec_motivacao}

Este documento é um \emph{template} que foi concebido, primariamente, para ser utilizado na redação de teses de doutorado, dissertações de mestrado, projetos de qualificação tanto de mestrado quanto de doutorado, escritos em português brasileiro (eventualmente, com partes em inglês) e em conformidade com as normas da ABNT.

Não obstante, ele também poderá ser utilizado, com ligeiras adaptações para a redação de outros trabalhos acadêmicos monográficos (e.g., trabalhos de conclusão de curso de graduação ou de especialização \emph{lato sensu}).

Antes de começar a escrever o seu trabalho acadêmico utilizando este \emph{template}, é bom saber que há um arquivo que você precisará editar para que a capa e a folha de rosto de seu trabalho sejam geradas.
Este arquivo é o \textbf{preambulo.tex} e se encontra no diretório \textbf{elementos-pre-textuais}.
Nesse arquivo, você deverá informar o seu nome, título do trabalho acadêmico, se o documento será uma tese de doutorado ou dissertação de mestrado ou projeto de qualificação, nome de seu(s) orientador(es), e outras informações necessárias.

Por fim, caso observe algum problema ou qualquer outro tipo de falha ou mal comportamento neste modelo, comunique-nos para que possamos tentar corrigi-los em futuras atualizações.

\section{Definição do problema de pesquisa}
\label{sec_definicao_problema_pesquisa}

Inserir seu texto aqui...

\section{Objetivos}
\label{sec_objetivos}

Inserir seu texto aqui...

\section{Contribuições}
\label{sec_contribuicoes}

Inserir seu texto aqui...

\section{Organização do trabalho}
\label{sec_organizacao_trabalho}

Normalmente ao final da introdução é apresentada, em um ou dois parágrafos curtos, a organização do restante do trabalho acadêmico.
Deve-se dizer o quê será apresentado em cada um dos demais capítulos.

Segue um exemplo:

Este trabalho está organizado em capítulos, incluindo o presente.
No \autoref{chap_fundamentacao_teorica} são apresentados alguns dos principais conceitos necessários que fundamentam o desenvolvimento deste trabalho.
A \hyperref[chap_trabalhos_relacionados]{revisão bibliográfica} deste trabalho apresenta uma revisão dos principais estudos relacionados ao tema, descrevendo seus resultados e suas contribuições.
Por fim, no \autoref{chap_conclusao} são apresentadas as conclusões, bem como as perspectivas de trabalhos futuros.
