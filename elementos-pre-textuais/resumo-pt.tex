% ------------------------------------------------------------------------------
% Resumo
% ------------------------------------------------------------------------------

\begin{resumo}
    Síntese do trabalho em texto cursivo contendo um único parágrafo.
    Para uma Tese de Doutorado o resumo deve conter, no máximo, 500 palavras.
    Para uma Dissertação de Mestrado o resumo deve conter, no máximo, 250 palavras.
    Para um Projeto de Qualificação o resumo deve conter, no máximo, 200 palavras.
    O resumo é a apresentação clara, concisa e seletiva do trabalho.
    No resumo deve-se incluir, preferencialmente, nesta ordem: breve introdução ao assunto do trabalho de pesquisa (incluindo motivação e justificativa para a realização deste trabalho), o que será feito no trabalho (objetivos), como ele será desenvolvido (metodologia), quais são os principais resultados obtidos ou esperados e a conclusão (compare os resultados com os da literatura e destaque as principais contribuições científicas do trabalho.

    \textbf{Palavras-chave}: Modelo Latex. Trabalho acadêmico monográfico. Normas ABNT. Outra palavra.
\end{resumo}

% ------------------------------------------------------------------------------
% Escolha de 3 a 6 palavras ou termos que descrevam bem o seu trabalho.
% As palavras-chaves são utilizadas para indexação. A letra inicial de cada
% palavra deve estar em maiúsculas. As palavras-chave são separadas por ponto.
% ------------------------------------------------------------------------------
