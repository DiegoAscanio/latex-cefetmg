%
% Documento: Introdução
%

\chapter{Introdução}\label{chap:introducao}

Este modelo prove um arquivo \textit{makefile}, portanto, para gerar este documento no formato PDF, basta apenas executar o comando {\ttfamily make} no Linux.
Para limpar os arquivos temporários, basta digitar o comando {\ttfamily make clean}.

Cada capítulo deve conter uma pequena introdução (tipicamente, um ou dois parágrafos) que deve deixar claro o objetivo e o que será discutido no capítulo, bem como a organização do capítulo.
Veja o exemplo abaixo.

A inclusão de reticências (\ldots) no texto deverá ser feita através de um comando especial denominado \verb|\ldots|.
Assim esse comando deverá ser utilizado ao invés da digitação de três pontos.

A introdução deverá apresentar uma visão de conjunto do trabalho a ser realizado, com o apoio da literatura, situando-o no contexto do estado da arte da área científica específica, sua relevância no contexto da área inserida e sua importância específica para o avanço do conhecimento.

Para melhor entendimento do uso do estilo de formatação, aconselha-se que o potencial usuário analise os comandos existentes no arquivo {\ttfamily main.tex} e os resultados obtidos no arquivo {\ttfamily main.pdf} depois do processamento pelo software LATEX + BIBTEX \cite{LaTeX2009,BibTeX2009}.
Recomenda-se a consulta ao material de referência do software para a sua correta utilização \cite{Lamport1986,Buerger1989,Kopka2003,Mittelbach2004}.

\section{Motivação}
\label{sec:motivacao}

O estilo de documento utilizado é o {\ttfamily abntex2}.
Através desse estilo a constituição do documento torna-se facilitada, uma vez que o mesmo possui comandos especiais para auxiliar a distribuição/definição das diversas partes constituintes do projeto.
Esse estilo é baseado nas normas da ABNT\index{ABNT}.
Maiores detalhes relacionados aos comandos existentes no estilo poderão ser adquiridos através da documentação disponível no site \href{https://code.google.com/p/abntex2/}{https://code.google.com/p/abntex2/} \cite{abntex2classe}.

Uma das principais vantagens do uso do estilo de formatação para LATEX é a formatação \textit{automática} dos elementos que compõem um documento acadêmico, tais como capa, folha de rosto, dedicatória, agradecimentos, epígrafe, resumo, abstract, listas de figuras, tabelas, siglas e símbolos, sumário, capítulos, referências, etc.
